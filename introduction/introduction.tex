======= OUTLINE =======

BACKGROUND
Survery of historical developments that inform dissertation
Write at a high school or early college biology level

I. Proteins are the molecular machines that drive biology
    A. Four major macromolecules and briefly discuss broad roles of each
        1. Nucleic acids
        2. Carbohydrates
        3. Lipids
        4. Protein
    B. Some major examples of proteins
        1. Hemoglobin and sickle cell
        2. Color blindness and opsins
        3. Lactase and lactose intolerance
II. Proteins are linear chains of amino acids with a three-dimensional structure
    A. Background on biochemistry, primary, secondary, and tertiary structure
    B. Structure-function relationships returning to previous hemoglobin example
    C. Structural biology seeks to determine protein structures to understand function
    D. Overview of structural methods including purification steps
        1. X-ray crystallography
        2. NMR
        3. cryo-EM
III. Disordered proteins do not adopt a stable structure
    A. Definition and discussion of protein disorder
        1. Structural heterogeneity, conformational ensembles, floppy
        2. Major sequence characteristics and structural "modes"?
    B. Disordered proteins are challenging to study with traditional methods
    C. Genome sequencing and computational tools revealed disordered proteins are ubiquitous
IV. Disordered proteins perform varied functions
    A. Flexibility confers several advantages
    A. Survey of disordered proteins and their functions
        1. Linkers
        2. Signaling
V. Disordered proteins are enriched in transcriptional machinery
    A. Introduce transcription via the central dogma
    B. Describe transcription factors, cofactors, transcriptional machinery
        1. Examples of different transcription factors
    C. Describe properties of transcription factors
        1. DNA binding and activation domains
        2. Different flavors of activation domains
    D. How transcription factors find their targets to initiate specific genetic programs is one of the major outstanding questions in molecular biology
VI. Disordered proteins are associated with phase separation
    A. Describe phase separation in general
    B. Describe phase separation as related to proteins
    C. Describe functions of phase separation and relationship to subcellular structures
VII. Phase separation via IDRs is a possible mechanism for initiating transcription
    A. Summarize Stadler's papers and others

AIMS
Discuss more recent developments in field and propose driving questions

I. Structure-function relationships of disordered proteins are unclear
    A. Examples of studies that have found specific cases of conservation in IDRs
    B. Alan Moses paper as reference in yeast
II. Use evolutionary analyses to identify conserved properties of disordered regions
    A. Conservation is powerful signal of function
        1. Examples of other studies that applied evolution
    B. Outline overall approach and make correspondence to chapters
        1. Identification of orthologous groups
        2. Application of evolutionary analyses

======= DRAFT =======

Life is a physical phenomenon. Despite the complexity of living things, their processes are governed by the same physical laws that describe how the planets revolve around the sun and the probability that an atom of uranium decays from one moment to another. However, as many systems are too complex to describe with physical models and equations, scientists simplify them into levels of abstraction that are more useful. (This is the origin of the common observation that biology is applied chemistry and chemistry is applied physics.) For example, Punnett squares facilitate the prediction of genotypes and phenotypes by distilling the complexities and nuances of diverse reproductive systems into a set of simple rules. However, since life spans a scale from single cells to entire ecosystems, biology likely employs more layers of abstraction than any other scientific discipline. One of the most powerful and widely used frameworks within the life sciences is biochemistry, which characterizes biological processes in terms of their component molecules and chemical reactions. A specific focus is four classes of macromolecules called nucleic acids, carbohydrates, lipids, and proteins, all of which are unique to biological systems. Though not all biological molecules are macromolecules and not all biological macromolecules fit neatly in one of these four categories, much of life at the molecular level is understood in terms of their structure and function. Each of the four has a characteristic role. Nucleic acids, i.e., DNA and RNA, are responsible for information storage and transfer. Carbohydrates primarily store energy but can also act as structural components of cells. Lipids are a diverse class of oily molecules which are components of cell membranes, store energy, and transmit signals. Proteins have a range of functions, including catalyzing reactions, transmitting signals, transporting materials, and providing structure. Many of these overlap with the functions of the other macromolecule classes because proteins are involved in virtually every biological process. However, unlike the others, which are often passive participants, proteins are highly active and dynamic. They respond to signals, change shape, and often use the other macromolecules as substrates in their activities. Proteins are essentially the molecular machines that carry out life's functions.

FOOTNOTE: Macromolecule is a loose term applied to molecules with high molecular mass. In practice, it usually refers to one of the classes listed above, among a few other prominent non-biological examples.

Some examples will illustrate the central role of proteins more clearly.
These examples should be accessible to a wide audience, so nothing too abstract. Ideally should relate to human physiology and disease.
    1. Hemoglobin and sickle cell
        A. Introduce red blood cells transport oxygen
        B. Hemoglobin is protein that binds oxygen.
            i. Physically interacts with a molecule of oxygen
        C. Sickle cell is disease caused by mutation in hemoglobin protein.
            i. Cells produce hemoglobin with small mistake which causes them to clump into chains
            ii. This distorts the shape of red blood cells into a sickle
            iii. Red blood cells do not flow as easily through blood vessels causing pain and oxygen deprivation
    2. Photopsins and color blindness
        A. Photopsins are light sensitive proteins in cells in the retina at the back of the eye
        B. Three different versions are sensitive to red, green, and blue light
        C. Color blindness is the result of cells missing one of these proteins
    3. Lactase and lactose intolerance
        A. Lactose is a sugar found in milk that requires specific protein, lactase, to digest
        B. Many humans who can digest milk products in childhood, lose this ability in adulthood suffer symptoms such as bloating and diarrhea after consuming dairy products
        C. They stop producing lactase, so the undigested sugars are broken down by bacteria in the colon

Despite performing this diverse range of functions, all proteins are made from of a set of 20 simple building blocks called amino acids. Though each amino acid is chemically unique, they share a common backbone composed of two distinct and complementary receptor and donor sites for chemical bonds. Thus, in a protein the amino acids are bonded in a linear chain like beads on a string. However, once synthesized, proteins are not tidy rod-shaped molecules. Instead, the chain loops and weaves between itself creating a characteristic three-dimensional structure in a process called folding. These structures, which are highly stable and characteristic of each protein, are a result of the interactions between the amino acids in the chain and the surrounding medium, which is typically water. Because each amino acid has unique geometric and chemical properties that influence the energetics of these interactions, a protein's three-dimensional structure is encoded by the sequence of amino acids that compose it. A protein's function is in turn a direct result of its structure. For example, the structure of hemoglobin precisely positions its amino acids and a helper molecule called a heme group to create a pocket that can stably but reversibly bind oxygen. This allows hemoglobin to carry oxygen throughout the body until it is delivered to its destination.

FOOTNOTE: The term amino acid encompasses any compound that contains an amino and carboxyl group. However, proteins are only synthesized from the twenty "canonical" amino acids. Another two (selenocysteine and pyrrolysine) are incorporated via a distinct mechanism under rare circumstances and are therefore considered non-standard.