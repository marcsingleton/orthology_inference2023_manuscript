======= OUTLINE =======

BACKGROUND
Survery of historical developments that inform dissertation
Write at a high school or early college biology level

I. Proteins are the molecular machines that drive biology
    A. Four major macromolecules and briefly discuss broad roles of each
        1. Nucleic acids
        2. Carbohydrates
        3. Lipids
        4. Protein
    B. Some major examples of proteins
        1. Hemoglobin and sickle cell
        2. Color blindness and opsins
        3. Lactase and lactose intolerance
II. Proteins are linear chains of amino acids with a three-dimensional structure
    A. Background on biochemistry, primary, secondary, and tertiary structure
    B. Structure-function relationships returning to previous hemoglobin example
    C. Structural biology seeks to determine protein structures to understand function
    D. Overview of structural methods including purification steps
        1. X-ray crystallography
        2. NMR
        3. cryo-EM
III. Disordered proteins do not adopt a stable structure
    A. Definition and discussion of protein disorder
        1. Structural heterogeneity, conformational ensembles, floppy
        2. Major sequence characteristics and structural "modes"?
    B. Disordered proteins are challenging to study with traditional methods
    C. Genome sequencing and computational tools revealed disordered proteins are ubiquitous
IV. Disordered proteins perform varied functions
    A. Flexibility confers several advantages
    A. Survey of disordered proteins and their functions
        1. Linkers
        2. Signaling
V. Disordered proteins are enriched in transcriptional machinery
    A. Introduce transcription via the central dogma
    B. Describe transcription factors, cofactors, transcriptional machinery
        1. Examples of different transcription factors
    C. Describe properties of transcription factors
        1. DNA binding and activation domains
        2. Different flavors of activation domains
    D. How transcription factors find their targets to initiate specific genetic programs is one of the major outstanding questions in molecular biology
VI. Disordered proteins are associated with phase separation
    A. Describe phase separation in general
    B. Describe phase separation as related to proteins
    C. Describe functions of phase separation and relationship to subcellular structures
VII. Phase separation via IDRs is a possible mechanism for initiating transcription
    A. Summarize Stadler's papers and others

AIMS
Discuss more recent developments in field and propose driving questions

I. Structure-function relationships of disordered proteins are unclear
    A. Examples of studies that have found specific cases of conservation in IDRs
    B. Alan Moses paper as reference in yeast
II. Use evolutionary analyses to identify conserved properties of disordered regions
    A. Conservation is powerful signal of function
        1. Examples of other studies that applied evolution
    B. Outline overall approach and make correspondence to chapters
        1. Identification of orthologous groups
        2. Application of evolutionary analyses

======= DRAFT =======

Life is a physical phenomenon. Despite the complexity of the cells, tissues, and organs that compose living things, all these processes are governed by the same physical laws that describe how the planets revolve around the sun and the probability that an atom of uranium decays from one moment to another. However, as many systems are too complex to describe with physical models and equations, scientists simplify to achieve levels of abstraction that are more useful. (This is the origin of the common observation that biology is applied chemistry and chemistry is applied physics.) Since life spans a scale from the size of single cell to entire ecosystems, biology likely employs more layers of abstraction than any other scientific discipline. One of the most powerful and widely used frameworks within the life sciences is biochemistry, which characterizes biological processes in terms of their component molecules. A specific focus is four classes of macromolecules called nucleic acids, carbohydrates, lipids, and proteins, all of which are unique to biological systems. Though not all biological molecules are macromolecules and not all biological macromolecules fit neatly in one of these four categories, much of life at the molecular level is understood in terms of their structure and function. Each of the four has a characteristic role. Nucleic acids, i.e. DNA and RNA, are responsible for information storage and transfer. Carbohydrates primarily store energy but can also act as structural components of cells. Lipids are a diverse class of oily molecules which are components of cell membranes, store energy, and transmit signals. Proteins have a diverse range of functions including but not limited to catalyzing reactions, transmitting signals, and providing structure. Proteins have such a diverse range of associated functions, many of which overlap with the other three categories, because proteins are involved in virtually every biological process. Moreover, proteins are highly dynamic. They respond to signals, change shape, and even use the other macromolecules as raw material for their activities. In this way, proteins are like molecular machines.

Macromolecule is a loose term applied to molecules with high relative molecular mass. In practice, it usually refers to one of the classes listed above, among a few other prominent non-biological examples.

